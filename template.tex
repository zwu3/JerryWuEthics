\documentclass[10pt,twocolumn]{article} 

\usepackage{oxycomps} % use the main oxycomps style file

\bibliography{references}

\pdfinfo{
    /Title (Ethics: Decentralized Finance)
    /Author (Zewei Wu)
}

\title{Ethics: Decentralized Finance}

\author{Zewei Wu}
\affiliation{Occidental College}
\email{zwu@oxy.edu}

\begin{document}

\maketitle

\begin{Ethics}
The current state of decentralized finance research encompasses a wide range of ethical challenges and situations, as well as a diversified user base. With the continued development of Bitcoin technology and its widespread adoption, it is feasible that this technology may become a global financial transaction mechanism. However, there are still a lot of unsolved questions. These decentralized peer-to-peer transactions raise numerous privacy and ethical concerns. Furthermore, the future of some cryptocurrencies has remained dubious.

Concerning the Blockchain's immutability, an important ethical point must be posed. From an operational standpoint, blockchain is, of course, a fantastic breakthrough. To have a limitless number of immutable transaction records, as well as a large amount of data to support each Ledger entry. Amazing and incomprehensible at the moment. The idea behind the "immutable model" is that having a better collection of "real-time data" will result in better forecasting of bitcoin trends. The removal of ambiguity in the Micro could lead to errors in the Macro. Is technology capable of truly bridging the valuation, accounting, and ledger gaps? Alternatively, the grand concept of Blockchain may be destined to be used solely for the distribution of information.

In DeFi, there is a "decentralization illusion," because the necessity for governance necessitates some level of centralisation, and structural characteristics of the system result in power concentration. DeFi's flaws could jeopardize financial stability if it becomes widely used. Due to excessive leverage, liquidity mismatches, built-in interconnection, and a lack of shock absorbers like banks, these can be severe. Existing governance processes in DeFi would serve as natural reference points for authorities when dealing with issues such as financial stability, investor protection, and illegal activities. Various sorts of intermediation support the cryptocurrency markets. DeFi, in theory, can be used to supplement traditional financial activity.
However, it currently has few real-world applications and is primarily used for speculation and arbitrage among several crypto assets. Given its self-contained character, DeFi-driven disruptions in the broader financial system and actual economy appear to be restricted for the time being. Concentration can enable collusion and hinder the viability of blockchains. It raises the possibility that a few major validators will gather enough authority to manipulate the blockchain for financial advantage. 

Furthermore, huge validators could clog the blockchain by conducting false trades between their own wallets, causing other traders' fees to skyrocket. Another fear is that validators may be able to foreclose huge orders in order to increase trading profits. Although front-running is also seen in traditional finance, it is frowned upon by regulators. Investors will be harmed by these rent-seeking actions, which may undermine DeFi's appeal in the future. Changes to governance protocols, specifically to prevent collusion, have gained traction in the DeFi community. Adopting these modifications, however, would not change the fact that some centralization is unavoidable.
While DeFi is still in its infancy, it provides services that are similar to those given by traditional finance and has some of the same flaws. The fundamental mechanisms that give birth to these vulnerabilities — leverage, liquidity mismatches, and their interaction via profit-seeking and risk-management techniques – are all well-known in the established financial system. However, several aspects of DeFi may make them particularly disruptive. We'll start with the role of leverage and run-risk in stablecoins due to liquidity mismatches in this section, then go on to spillover channels to traditional intermediaries.

The DeFi ecosystem is continuously evolving, despite its tremendous growth. It is now focused mostly for crypto asset speculation, investment, and arbitrage, rather than real-economy use cases. DeFi is vulnerable to criminal activities and market manipulation due to the insufficient application of anti-money laundering and transaction anonymity. Overall, DeFi's core premise – lowering the rents paid to centralized middlemen – does not appear to have been accomplished.
History has shown that the early development of novel technologies is typically accompanied by bubbles and a loss of market integrity, despite the fact that it produces inventions that could be useful to a wider audience in the future. DeFi might still play a key role in the financial system with advancements to blockchain scalability, large-scale tokenization of traditional assets, and, most critically, appropriate regulation to maintain protections and boost confidence.

The link between blockchain and artificial intelligence is changing. Blockchain is a record-keeping system that allows for decentralized decision-making and unchangeable records. However, AI may be used to augment the data that is input into the blockchain system and eventually printed into the blocks of a blockchain. To put it another way, AI is the link between blocks and external data. Consider a blockchain-based social network with a distributed ledger. Blocks contain all of a user's public information. New blocks, on the other hand, can be added using AI algorithms that collect user data and store it in new blocks.

The distributed social network differs from the present one in that AI algorithms choose which information is shared among users and added to new blocks, allowing for increased decentralization. It's also possible that the AI algorithm and the types of data captured can be changed based on the consensus procedures that serve as the social network's underlying basic logic. Because blockchain's immutable data storage capabilities meets AI's ever-increasing ability to generate data, the usage of blockchain in conjunction with AI raises new ethical problems.

Smart contract features ensure transaction accountability, transparency, and speed while reducing the risk of a breach. Smart contracting, on the other hand, eliminates the potential of negotiating and negotiation\cite{YTangEthics}. Some consider negotiating to be moral because it establishes a relationship between the promisor and the promisee and promotes relational justice. As a result, automation of responsibilities may have a negative impact on human dignity by reducing the number of points of entry into talks and bargaining.

A alternative ethical perspective is provided by focusing on the ramifications of smart contracts. The use of smart contracts for compensation enactment has the benefit of increasing efficiency. Recipients will be compensated appropriately after completing a task with simplified payments. Users will be paid automatically once their work is completed. Users can rest assured that their work will be fairly compensated due to the difficulty of breaching a contract. Furthermore, because the terms of the agreement are immutable, smart contracting encourages participants to add objective preset criteria. Transparency and accountability are promoted in this manner. The risk of incompleteness in computer codes will also encourage users to specify their relationships with greater precision before using an effective dispute resolution system.

Smart contracts formalize and automate the performance obligation. Smart contracts are likely to be preferred by deontologists because the rules are specified prior to the start of any employment contract. This reduces the risk of a breach and raises the likelihood that parties will follow through on their promissory obligations. Individuals are likely to suffer harm as a result of smart contracts, especially if the pre-defined terms are one-sided.

However, this raises the same ethical concerns as today's boilerplate agreements. Furthermore, because the terms of a smart contract must be explicitly defined in order for it to be self-executing, the likelihood of confusing phrases being used is reduced. Furthermore, as detailed below, the idea of shared leadership through blockchain gives employees additional bargaining power, allowing them to include more equitable terms in smart contracts.

Contractarians would seek equity in compensation distribution. Employees will be treated similarly in that their salary would be distributed via smart contracts. The least fortunate would be paid in the same way as the rest of the workforce. Payment would be issued automatically to each employee. Employees, for example, could be given payment options by the organization, such as a mailed check or a direct deposit, and so on. After a method is chosen, smart contracts will issue payments once the work is completed.




   \end{ethics}


\end{document}